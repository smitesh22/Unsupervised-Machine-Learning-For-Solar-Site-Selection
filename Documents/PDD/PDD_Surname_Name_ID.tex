\documentclass[11pt]{article}
\usepackage[margin=0.4in]{geometry}
\usepackage[all]{xy}

\usepackage{amsmath,amsthm,amssymb,color,latexsym,graphicx}
\usepackage{geometry}        
\geometry{a4paper, top=1cm}    

\usepackage[scaled]{helvet}
\renewcommand\familydefault{\sfdefault} 
\usepackage[T1]{fontenc}

\usepackage{tabto}

\usepackage[numbers]{natbib}

\linespread{1.0}

\pagenumbering{gobble}

\newcommand{\sect}[1]{\begin{center}\textbf{#1}\end{center}}

\begin{document}
	
\noindent Student name: Smitesh Nitin Patil \tabto{8cm} Supervisor contact name: Dr. Karl Mason \\
\noindent Student number: 22223696 \tabto{8cm} Supervisor contact email: karl.mason@universityofgalway.ie 
\vspace{0.2cm}

\sect{Uncovering Optimal Solar Site Locations in India using Unsupervised Learning Approaches}

\sect{Introduction}
The process of transitioning the global energy supply from fossil fuel-based sources to
sustainable energy sources like wind and solar will be crucial 
for mankind to make in the 21st century. Solar energy is generated from photovoltaic
cells which need a high amount of solar irradiance throughout the year to be profitable.
Countries in tropical regions like parts of India tend to receive abundant sunlight throughout 
the year. However, many aspects need to be studied before identifying promising regions where
solar farms could be built to tap into that region's solar potential, like the slope gradient of
the terrain, proximity to urban centers, and nature, wildlife preserve areas.
\sect{Background Research or Context}
There are several challenges in identifying and classifying sites favorable for solar energy
 predictions from data gathering to methodology for identification. Studies for solar site 
selections have been carried out by Colak et al.\cite{colak_memisoglu_gercek_2020} and 
Al Garni et al.\cite{al_garni_awasthi_2017} for provinces in Turkey and Saudia Arabia respectively. 
Both authors used the Analytical Hierarch Process(AHP) a Multi-Criteria Decision Making(MCDM)
 the process where they would rank various aspects of geographical features and generate weights for 
the importance of aspects or features. Other MCDMs processes like the Fuzzy Logic model for site 
selection done by Zoghi et al.\cite{zoghi_houshang_ehsani_sadat_javad_amiri_karimi_2017} in Iran. 
Finally, using GIS data and carrying out AHP for identifying solar and wind power sites in India was done 
by Saraswat et al. \cite{saraswat_digalwar_yadav_kumar_2021} in 2021.
\sect{Proposed Project}
There has been a lot of research in this domain but all the studies more or less have used Multi-Criteria
Decision-making processes to evaluate GIS information and make suggestions.
This study aims to develop a novel unsupervised model for clustering potential sites
for solar farms. Firstly, the gathering of GIS(geographical information systems) 
data for multiple layers of information like terrain type, solar irradiance, wildlife protected sanctuaries.
Using GIS data to identify sites of concern has been done for studies like detecting landslide occurrences have 
been done by Chang et al. \cite{chang_du_zhang_huang_chen_li_guo_2020} using unsupervised learning approach,
but it has been relatively absent in the identification of areas suitable for renewable energy farms.
Secondly, after data gathering, it would be trained using clustering algorithms to identify 
sites based on selected features. Lastly, In the previous studies, it was hard to quantify the robustness 
and accuracy of the developed models, but here we can use unsupervised learning evaluation metrics 
like purity, and residual sum squares to evaluate our results. Additionally, auto-labeling of data 
could also be tried with set criteria, and then robust supervised learning methods could be used
to develop models on the labeled data. 
\sect{Timeline}
\begin{itemize}
    \setlength\itemsep{0em}
    \item Literature review - April 31, 2023
    \item Project Proposal - June 30, 2023
    \item Project Implementation - July 30, 2023
    \item Thesis writeup - August 20, 2023
    \item Project Completion - August 31, 2023
\end{itemize}
\vspace{-\baselineskip}
\vspace{-\baselineskip}
\bibliographystyle{IEEEtranN}
\renewcommand{\refname}{{\normalsize \sect{References}}}
\bibliography{PDD_references}
\end{document}